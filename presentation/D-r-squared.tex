\documentclass[presentation.tex]{subfiles} 

\begin{document}
	
	\begin{frame}
		\frametitle{D - $R^2$ }
		\begin{itemize}
			\item koeficient determinácie
			\item hodnota z intervalu [0, 1]
			\item ako dobre aproximuje regresný model hodnoty závislých premenných
			\item čím bližšie k 1, tým presnejší
		\end{itemize}
		
		
		

	\end{frame}

	\begin{frame}
		\frametitle{D - $R^2$ }
		\begin{columns}
			\begin{column}{0.5\textwidth}
				\centering
				\begin{align*}
						&R^2 = 1 - \frac{\sum_{i=1}^{n} (y_i - \hat{y}_i)^2}{\sum_{i=1}^{n} (y_i - \bar{y})^2}
				\end{align*}
			\end{column}
			\begin{column}{0.5\textwidth}
				\begin{itemize}
					\item rozdiely medzi skutočnými hodnotami $y$ a predpovedanými
					\item  rozdiely medzi skutočnými hodnotami $y$ a priemerom
				\end{itemize}
				
			\end{column}
		\end{columns}
	\end{frame}

	\begin{frame}
		\frametitle{D - $R^2$ }
				\begin{itemize}
					\item funkcia \pyth|r_squared(x, y, beta)| pomocou \pyth|numPy| 
					\item použijeme na dátach A04wine.csv
					\item získame koeficienty pre obe normy: \begin{align*} R^{2}_{(1)} &\approx  0.78813\\
															     R^{2}_{(\infty)} &\approx 0.80649
													\end{align*}
				\end{itemize}
			
	\end{frame}
	
\end{document} 