\documentclass[presentation.tex]{subfiles} 

\begin{document}
	
	\begin{frame}
		\frametitle{$R^2$ -- koeficient determinácie }
		\begin{itemize}
			\item typicky hodnota z intervalu [0; 1]
			\item podiel rozptylu závislej premennej zachytený modelom
			\item čím bližšie k 1, tým lepšie vysvetľuje rozptyl
		\end{itemize}
		
		
		

	\end{frame}

	\begin{frame}
		\frametitle{$R^2$ -- koeficient determinácie}
		\begin{columns}
			\begin{column}{0.5\textwidth}
				\centering
				\begin{align*}
						&R^2 = 1 - \frac{\sum_{i=1}^{n} (y_i - \hat{y}_i)^2}{\sum_{i=1}^{n} (y_i - \bar{y})^2}
				\end{align*}
			\end{column}
			\begin{column}{0.5\textwidth}
				\begin{itemize}
					\item rozdiely medzi skutočnými hodnotami $y$ a predpovedanými
					\item  rozdiely medzi skutočnými hodnotami $y$ a priemerom (rozptyl)
				\end{itemize}
				
			\end{column}
		\end{columns}

		\begin{columns}
			\begin{column}{\textwidth}
			\begin{itemize}
				\item ukazuje, aký podiel rozptylu závislej premennej je vysvetlený nezávislými premennými.
			\end{itemize}
			\end{column}
		\end{columns}
	\end{frame}

	\begin{frame}
		\frametitle{Výsledky pre naše predikcie}
				\begin{itemize}
					\item regresie $L^1$, $L^{\infty}$
					\item koeficienty pre obe normy: 
					\begin{align*} 
						R^{2}_{(1)} &\approx  0.78813\\
						R^{2}_{(\infty)} &\approx 0.80649
					\end{align*}
					\item obe dostatočne zachytávajú rozptyl
				\end{itemize}
			
	\end{frame}
	
\end{document} 