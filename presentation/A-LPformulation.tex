\documentclass[presentation.tex]{subfiles} 
\newcommand*{\vertbar}{\rule[0ex]{0.3pt}{1.5ex}}
\newcommand*{\horzbar}{\rule[0ex]{1.5ex}{0.3pt}}
\begin{document}
	
	\begin{frame}
		\frametitle{Formulácia úloh lineárneho programovania}
		\framesubtitle{Úloha}
		   Nájsť koeficienty $\beta_0$, $\beta_1$, $\ldots$, $\beta_k$ tak, aby predikovaný vektor 
		\begin{equation}
			\hat{y} = \beta_0 + \beta_1 x_1 + \ldots + \beta_k x_k
		\end{equation}
		bol čo najbližšie k výstupu $y$, kde $y$ označuje závislú premennú a $x_1$, $x_2$, \ldots, $x_k$ $\in$ $\mathbb{R}^n$ označujú nezávislé premenné.
		Túto vzdialenosť |y - $\hat{y}$| sme minimalizovali $l_1$ a $l_{\infty}$ normami
	\end{frame}
	
	\begin{frame}
		\frametitle{Minimalizovanie $l_1$ normy}
		Chceme minimalizovať normu $||y - \hat{y}||_1$ \\
		\text{označíme: }
		\begin{equation}
			\begin{split}
			\mathbf{A} := (1_n, x_1, \dots, x_k) \\
			\beta := (\beta_0, \beta_1, \dots, \beta_k)^T
			\end{split}
		\end{equation}
		\text Problém prevedieme do tvaru:
		\begin{align*}
			\text{min}~ &c^Tx\\
			&Ax \geq b
		\end{align*}
	\end{frame}
	
	\begin{frame}
		\text Zavedieme nový vektor $t$ $\in$ $\mathbb{R}^n$, ktorým ohraničíme $y - \mathbf{A} \beta$
		Minimalizovanie $l_{1}$ normy ako úloha lineárneho programovania:
		\begin{align*}
			\text{min}~ &
			\left(
			\begin{array}{c|c}
				\mathbf{0}_{k+1}^T & \mathbf{1}_n^T
			\end{array}
			\right)
			\left(
			\begin{array}{c}
				\beta \\
				\hline
				t
			\end{array}
			\right) \\
			&\left(
			\begin{array}{c|c}
				\mathbf{A} & \mathbb{I}_n \\
				\hline
				-\mathbf{A} & \mathbb{I}_n
			\end{array}
			\right)
			\left(
			\begin{array}{c}
				\beta \\
				\hline
				t
			\end{array}
			\right)
			\geq
			\left(
			\begin{array}{c}
				y \\
				\hline
				-y
			\end{array}
			\right) \\
			&\beta \in \mathbb{R}^{k+1},~t \geq \mathbf{0}_{n}
		\end{align*}
	\end{frame}
	
	\begin{frame}[shrink=0.5]
		\frametitle{Minimalizovanie  $l_{\infty}$ normy}
		\text Chceme minimalizovať normu $||y - \hat{y}||_{\infty}$ \\
		\text Zavedieme skalárnu premennú $\gamma \in \mathbb{R}$, prevedieme na úlohu LP 
		\begin{align*}
			-\gamma \mathbf{1}_n \leq y - \mathbf{A} \beta &\leq \gamma \mathbf{1}_n
		\end{align*}
		\text Pomocou značenia z (2), výsledná úloha:
		\begin{align*}
			\text{min}~ &
			\left(
			\begin{array}{c|c}
				\mathbf{0}_{k+1}^T & 1
			\end{array}
			\right)
			\left(
			\begin{array}{c}
				\beta \\
				\hline
				\gamma
			\end{array}
			\right) \\
			&\left(
			\begin{array}{c|c}
				\mathbf{A} & \mathbf{1}_n \\
				\hline
				-\mathbf{A} & \mathbf{1}_n
			\end{array}
			\right)
			\left(
			\begin{array}{c}
				\beta \\
				\hline
				\gamma
			\end{array}
			\right) 
			\geq
			\left(
			\begin{array}{c}
				y \\
				\hline
				-y
			\end{array}
			\right) \\
			&\beta \in \mathbb{R}^{k+1},~\gamma \geq 0 
		\end{align*}
	\end{frame}
\end{document}