\documentclass[report.tex]{subfiles}

\begin{document}
	
	\section{Úvod}	
	
	V našej práci sa budeme venovať implementácii lineárnej regresie ako úlohy lineárneho programovania. Lineárna regresia je spôsob odhadovania závislej premennej $y \in \mathbb{R}^n$ ako lineárnej kombinácie nezávislých premenných $x_1,\dots,x_k \in \mathbb{R}^n$ s pridaným skalárnym členom: $\hat{y} = \beta_0 + \beta_1x_1 + \dots + \beta_kx_k$. Takýto problém môžeme interpretovať ako $n$ pozorovaní, kde pre každé pozorovanie sledujeme $k$ atribútov, čiže vektor $x_i$ pre $i = 1,\dots ,k$ predstavuje dáta atribútu $i$ pre všetkých $n$ pozorovaní. Pomocou lineárnej funkcie týchto premenných sa budeme snažiť čo najlepšie predikovať atribút $y$.
	
	Na meranie vzdialenosti medzi vektorom $y$ a predikovaným vektorom $\hat{y}$ budeme používať $L^1$ a $L^{\infty}$ normy, keďže práve pre tie sa dá tento problém naformulovať ako úloha lineárneho programovania. V kapitole \ref{sec:A} sa venujeme matematickej formulácii LP úlohy a dokazovaniu jej optimality. V kapitole \ref{sec:B} vizualizujeme funkčnosť modelu na arbitrárnych 2D dátach \verb|A04plotregres.npz|. Následne, v kapitole \ref{sec:C} sa venujeme predikovaniu ceny vína podľa dátového súboru \verb|A04wine.csv|. Pre tieto predikcie následne spočítame koeficient determinácie v \ref{sec:D}. Na záver, sekcia \ref{sec:E} popisuje našu implementáciu $L^1$ a $L^{\infty}$ lineárnej regresie pre ľubovoľné dáta v programovacom jazyku \verb|Python|. Tiež sa tam venujeme porovnávaniu správania takýchto regresii a formulácii a implementácii minimalizovania váženej sumy týchto noriem.
	
\end{document}