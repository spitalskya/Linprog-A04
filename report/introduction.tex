\documentclass[report.tex]{subfiles}

\begin{document}
	
	\section{Úvod}	
	
	V našej práci sa budeme venovať implementácii lineárnej regresie ako úlohy lineárneho programovania. Lineárna regresia je spôsob odhadovania závislej premennej $y \in \mathbb{R}^n$ ako afinnej kombinácie nezávislých premenných $x_1,\dots,x_n \in \mathbb{R}^n$. Na meranie vzdialenosti medzi vektorom $y$ a afinnej kombinácie budeme moužívať $L^1$ a $L^{\infty}$ normy, keďže práve pre tie sa dá tento problém naformulovať ako LP úloha.
	
	V kapitole \ref{sec:A} sa venujeme matematickej formulácii LP úlohy a dokazovaniu jej optimality. V kapitole \ref{sec:B} vizualizujeme funkčnosť modelu na arbitrárnych 2D dátach \verb|A04plotregres.npz|. Následne, v kapitole \ref{sec:C} sa venujeme predikovaniu ceny vína podľa dátového súboru \verb|A04wine.csv|. Pre tieto predikcie následne spočítame koeficient determinácie v \ref{sec:D}. Na záver, sekcia \ref{sec:E} popisuje našu implementáciu $L^1$ a $L^{\infty}$ regresii pre ľubovoľné dáta v programovacom jazyku \verb|Python|. Tiež sa tam venujeme porovnávaniu správania takýchto regresii a formulácii a implementácii minimalizovania váženej sumy týchto noriem.
	
\end{document}