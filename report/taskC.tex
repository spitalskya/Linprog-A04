\documentclass[report.tex]{subfiles}

\begin{document}   

\section{Predikcia kvality vína}\label{sec:C}
V tejto úlohe sa snažíme predikovať kvalitu vína, inšpirovaní prístupom Orleya Ashenfeltera k predikcii cien vína z Bordeaux.

Využívame dáta zo súboru \verb|A04wine.csv| a aplikujeme modely $L^1$ a $L^{\infty}$ z úlohy A. Budeme využívať podobný postup ako v úlohe B. Na implementáciu formulovaných LP úloh využívame:
\begin{itemize}
	\item \pyth|pandas| - načítanie dát z \verb|csv| súboru
	\item \pyth|numpy| - tvorenie matíc a vektorov
	\item \pyth|scipy.optimize| - implementovaný LP solver
\end{itemize}

Vyberieme z dát dané nezávislé premenné \pyth|x| a závislú premennú \pyth|y|:

\begin{python}
y = data['Price']
x = data[['WinterRain','AGST', 'HarvestRain', 'Age', 'FrancePop']]
\end{python}

Z počtu nezávislých premenných získame rozmer vektora $\beta$ (+1 kvôli konštantnému členu):

\begin{python}
numberOfVariablesBeta = x.shape[1] + 1
\end{python}

Vytvoríme potrebné štruktúry pre zostavenie modelu normy $L^1$:

\begin{python}
c = np.array([0]*numberOfVariablesBeta + [1] * len(x.values)) # Objective function coefficients

ALeft = np.matrix([ [1] * len(x.values)]).transpose() # Coefficients for beta0
ARigth = np.matrix(x.values) # Coefficients for other independent variables beta
A = np.block([ALeft, ARigth]) # Concenate coefficients of variables into one matrix

I = np.identity(len(x.values)) # Identity matrix for intercept term
\end{python}

Naformulujeme problém a vyriešime pomocou \pyth|scipy.optimize.linprog|

\begin{python}
A_ub = np.block([[-A, -I], [A, -I]])
b_ub = np.concatenate([-y, y])

solve = linprog(c, A_ub, b_ub, bounds = [(None,None)]*numberOfVariablesBeta +[(0, None)])
\end{python}

Po vyriešení vyberieme z riešenia koeficienty:

\begin{python}
betas = solve.x[:numberOfVariablesBeta]
\end{python}

Čo nám dá: 

\begin{gather*}
	\beta_0^{(1)} \approx -8.8801 \cdot 10^{-1} ,~\beta_1^{(1)} \approx 1.5793\cdot 10^{-3},~\beta_2^{(1)} \approx 5.2130\cdot 10^{-1} \\
	\beta_3^{(1)} \approx -4.5137\cdot 10^{-3} ,~\beta_4^{(1)} \approx 1.1300\cdot 10^{-2}  ,~\beta_5^{(1)} \approx -2.2111\cdot 10^{-5}
\end{gather*}

Z týchto výsledkov môžeme usúdiť, že najviac pozitívne vplýva na cenu vína metrika \textit{AGST - Average growing season temperature} a najsignifikantnejší negatívny vplyv má \textit{dážď počas zberu}.


\newpage

Ďalej zostrojíme relevantné štruktúry a naformulujeme LP pre $L^{\infty}$ normu:

\begin{python}
c_inf = np.array([0]*numberOfVariablesBeta + [1]) # Koeficienty cielovej funkcie
A_inf = np.block([ALeft, ARigth]) # Koeficienty pre nezavisle premenne pre l_inf normu
i_inf = np.array([ [1] * len(x.values)]).transpose() # Koeficienty pre intercept term pre l_inf normu

A_ub_inf = np.block([[-A_inf,-i_inf],[A_inf,-i_inf]])
b_ub_inf = np.concatenate([-y, y]) 
\end{python}

Vyriešime aj tento problém pomocou \pyth|scipy.optimize.linprog()| pre $L^{\infty}$ normu a vyberieme $\beta$ koeficienty:

\begin{python}
solve_inf = linprog(c_inf, A_ub_inf, b_ub_inf, bounds=[(None,None)]*numberOfVariablesBeta+[(0,None)])
betas_inf = solve_inf.x[:numberOfVariablesBeta]
\end{python}

Po čom dostaneme:

\begin{gather*}
	\beta_0^{(\infty)} \approx  3.4841 ,~\beta_1^{(\infty)} \approx 8.3399\cdot 10^{-4} ,~\beta_2^{(\infty)} \approx 6.0027\cdot 10^{-1} \\
	\beta_3^{(\infty)} \approx -3.3416\cdot 10^{-3} ,~\beta_4^{(\infty)} \approx -2.3036\cdot 10^{-2}  ,~\beta_5^{(\infty)} \approx -1.1958\cdot 10^{-4}
\end{gather*}

Vidíme, že aj lineárna regresia pomocou $L^{\infty}$ normy odhaduje najväčší pozitívny vplyv metriky \textit{AGST} a najväčší negatívny vplyv \textit{dažďu počas zberu}. Zmenil sa však vplyv premennej \textit{vek} (oproti prechádzajúcemu modelu) z pozitívneho na negatívny.


\end{document}
