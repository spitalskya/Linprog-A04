\documentclass[report.tex]{subfiles}

\begin{document}   

\section{Predikcia kvality vína}\label{sec:C}
V tejto úlohe sa snažíme predikovať kvalitu vína, inšpirovaní prístupom Orleya Ashenfeltera k predikcii cien vína z Bordeaux.

Využívame dáta zo súboru \verb|A04wine.csv| a aplikujeme modely $L^1$ a $L^{\infty}$ lineárnej regresie z úlohy \ref{sec:A}. Budeme využívať podobný postup ako v úlohe \ref{sec:B}. Na implementáciu formulovaných LP úloh využívame:
\begin{itemize}
	\item \pyth|pandas| - načítanie dát z \verb|csv| súboru
	\item \pyth|numpy| - tvorenie matíc a vektorov
	\item \pyth|scipy.optimize| - implementovaný LP solver
\end{itemize}

Vyberieme z dát dané nezávislé premenné \pyth|x| a závislú premennú \pyth|y|:

\begin{python}
y = data['Price']
x = data[['WinterRain','AGST', 'HarvestRain', 'Age', 'FrancePop']]
# Calculate the number of variables (features)
k = x.shape[1]
\end{python}

Vytvoríme potrebné štruktúry pre zostavenie modelu normy $L^1$:

\begin{python}
c = np.concatenate(([0]*(k + 1), np.ones(len(x.values)))) # Objective function coefficients (plus 1 for the intercept term)
A = np.block([np.ones((len(x.values), 1)), np.array(x.values)]) # Concatenate coefficients of variables into one matrix
\end{python}

Naformulujeme problém a vyriešime pomocou \pyth|scipy.optimize.linprog|:

\begin{python}
# Formulate inequality constraints for L1 norm
A_ub = np.block([[-A, -I], [A, -I]])
b_ub = np.concatenate([-y, y])
bounds = [(None, None)]*(k + 1) +[(0, None)] * len(x.values)

solve = linprog(c, A_ub, b_ub, bounds=bounds)
\end{python}

Po vyriešení vyberieme z riešenia koeficienty, čo nám dá: 

\begin{gather*}
	\beta_0^{(1)} \approx -8.8801 \cdot 10^{-1} ,~\beta_1^{(1)} \approx 1.5793\cdot 10^{-3},~\beta_2^{(1)} \approx 5.2130\cdot 10^{-1} \\
	\beta_3^{(1)} \approx -4.5137\cdot 10^{-3} ,~\beta_4^{(1)} \approx 1.1300\cdot 10^{-2}  ,~\beta_5^{(1)} \approx -2.2111\cdot 10^{-5}
\end{gather*}

Z týchto výsledkov môžeme usúdiť, že najviac pozitívne vplýva na cenu vína metrika \textit{AGST - Average growing season temperature} a najsignifikantnejší negatívny vplyv má \textit{dážď počas zberu}.


Ďalej zostrojíme relevantné štruktúry a naformulujeme LP pre $L^{\infty}$ normu:

\begin{python}
c_inf = np.concatenate(([0]*(k + 1), [1]))
A_inf = np.block([np.ones((len(x.values), 1)), np.array(x.values)]) # Coefficients for independent variables for L-inf norm
i_inf = np.ones((len(x.values), 1)) # Coefficients for gamma scalar variable

# Formulate inequality constraints for L-inf norm
A_ub_inf = np.block([[-A_inf, -i_inf], [A_inf, -i_inf]])
b_ub_inf = np.concatenate([-y, y])
bounds_inf = [(None, None)]*(k + 1) + [(0, None)] 
\end{python}

\newpage

Vyriešime aj tento problém pomocou \pyth|scipy.optimize.linprog()| pre $L^{\infty}$ normu a vyberieme $\beta$ koeficienty:

\begin{python}
solve_inf = linprog(c_inf, A_ub_inf, b_ub_inf, bounds=bounds_inf)
\end{python}

\begin{gather*}
	\beta_0^{(\infty)} \approx  3.4841 ,~\beta_1^{(\infty)} \approx 8.3399\cdot 10^{-4} ,~\beta_2^{(\infty)} \approx 6.0027\cdot 10^{-1} \\
	\beta_3^{(\infty)} \approx -3.3416\cdot 10^{-3} ,~\beta_4^{(\infty)} \approx -2.3036\cdot 10^{-2}  ,~\beta_5^{(\infty)} \approx -1.1958\cdot 10^{-4}
\end{gather*}

Vidíme, že aj lineárna regresia pomocou $L^{\infty}$ normy odhaduje najväčší pozitívny vplyv metriky \textit{AGST} a najväčší negatívny vplyv \textit{dažďu počas zberu}. Zmenil sa však vplyv premennej \textit{vek} (oproti prechádzajúcemu modelu) z pozitívneho na negatívny.


\end{document}
