\documentclass[12pt,oneside,a4paper,slovak]{article}           

\usepackage{latexsym}
\usepackage{amsfonts}
\usepackage{amsthm}
\usepackage{amstext}
\usepackage{amsmath}
\usepackage{amscd}
\usepackage{amssymb}
\usepackage{graphicx}
\usepackage{titlesec}
\usepackage{listings}
\usepackage{pythonhighlight}
\lstset{language=Python}
\usepackage[skip=0pt]{caption}
\usepackage{geometry}
\usepackage[final]{hyperref}
\usepackage[T1]{fontenc}                % slovencina
\usepackage[utf8]{inputenc}
\usepackage[slovak]{babel}

\geometry{top=2cm, bottom=2cm, left=2.5cm, right=2.5cm, headheight=0.6cm, headsep=0.3cm, footskip=1cm}

\renewcommand{\rmdefault}{phv} % Arial
\renewcommand{\sfdefault}{phv} % Arial
\renewcommand{\baselinestretch}{1}
\newcommand*{\vertbar}{\rule[0ex]{0.3pt}{1.5ex}}
\newcommand*{\horzbar}{\rule[0ex]{1.5ex}{0.3pt}}

\begin{document}

\section*{Úloha E}	

\subsection*{Spracovanie všeobecnej triedy pre $L^1$ a $L^{\infty}$ regresiu}

Vypracovali sme modul \pyth|Model| pre počítanie $L^1$ a $L^{\infty}$ regresie z ľubovoľných číselných dát, ktorý využíva LP formulácie popísané vyššie. Konkrétne \pyth|L1Model| využíva formuláciu na minimalizovanie $L^1$ normy a \pyth|LInfModel| minimalizuje $L^{\infty}$ normu. Príklad použitia tohto modelu sa nachádza v \verb|model_demonstration.ipynb| Následne opíšeme jednotlivé metódy jednotlivých modelov.

\subsubsection*{\pyth|Model(dependent_vect, independent_vect)|}

Konštruktor triedy, spoločný pre oba modely, vytvorí inštanciu, ktorá si drží dáta a vie na nich vykonávať operácie popísané nižšie. 

Argumenty:

\begin{itemize}
	\item \pyth|dependent_vect: np.array| - vektor závislých premenných
	\item \pyth|independent_vect: np.array| - matica, ktorej stĺpce sú vektory nezávislých premenných
\end{itemize}

\subsubsection*{\pyth|Model.solve()|}

Metóda, ktorá vyrieši regresnú LP úlohu na daných dátach. \pyth|L1Model.solve()| rieši minimalizáciou $L^1$ normy a \pyth|LInfModel.solve()|, rieši minimalizáciou $L^{\infty}$ normy. 

Vracia:

\begin{itemize}
	\item \pyth|np.array| - vektor optimálnych $\beta$ premenných
\end{itemize}

Po zavolaní tejto metódy si inštancia uloží vektor optimálnych $\beta$ premenných do atribútu \pyth|self._beta|, potrebné pre metódy popísané nižšie.

\subsubsection*{\pyth|Model.r2()|}

Vypočíta $R^2$ koeficient pre dané dáta a vypočítaný vektor $\beta$.

Vracia:

\begin{itemize}
	\item \pyth|float| - výsledný $R^2$ koeficient
\end{itemize}

\subsubsection*{\pyth|Model.visualize()|}

Ak je počet nezávislých premenných 1 alebo 2, táto metóda vykreslí graf dát spolu s vypočítanou regresnou priamkou, resp. rovinou. 

Vracia:

\begin{itemize}
	\item \pyth|bool| - úspešnosť vizualizácie, kde \pyth|False| označuje, že nezávislých premenných je viac ako 2, čiže nie je možné vykresliť graf
\end{itemize}

\end{document}