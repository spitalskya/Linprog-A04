\documentclass[12pt,oneside,a4paper,slovak]{article}           

\usepackage{latexsym}
\usepackage{amsfonts}
\usepackage{amsthm}
\usepackage{amstext}
\usepackage{amsmath}
\usepackage{amscd}
\usepackage{amssymb}
\usepackage{graphicx}
\usepackage{titlesec}
\usepackage{listings}
\usepackage{pythonhighlight}
\usepackage[skip=0pt]{caption}
\usepackage{geometry}
\usepackage[final]{hyperref}
\usepackage[T1]{fontenc}                % slovencina
\usepackage[utf8]{inputenc}
\usepackage[slovak]{babel}

\geometry{top=2cm, bottom=2cm, left=2.5cm, right=2.5cm, headheight=0.6cm, headsep=0.3cm, footskip=1cm}

\renewcommand{\rmdefault}{phv} % Arial
\renewcommand{\sfdefault}{phv} % Arial
\renewcommand{\baselinestretch}{1}
\newcommand*{\vertbar}{\rule[0ex]{0.3pt}{1.5ex}}
\newcommand*{\horzbar}{\rule[0ex]{1.5ex}{0.3pt}}


\begin{document}   

\section*{Úloha D}
Vytvorme funkciu \pyth|r_squared(x, y, beta)| - kde \pyth|x| je matica vektorov nezávislých premenných, \pyth|y| je vektor závislej premennej, \pyth|beta| je vektor optimálnych $\beta$ koeficientov získaných regresiou - ktorá bude počítať $R^2$ koeficient podľa definície:

\begin{align*}
	&R^2 = 1 - \frac{\sum_{i=1}^{n} (y_i - \hat{y}_i)^2}{\sum_{i=1}^{n} (y_i - \bar{y})^2} &\hat{y} = \beta_0 + \beta_1x_1 + \dots + \beta_kx_k,~\bar{y} = \frac{1}{n} \sum_{i=1}^ny_i
\end{align*}

\begin{python}
def r_squared(x: np.ndarray, y: np.ndarray, beta: np.ndarray) -> float:
	# calculate y-hat and mean of y vector
	y_hat = beta[0] + np.dot(x, beta[1:])
	y_mean = np.mean(y)
	
	res1 = 0    # partial result for the numerator in the formula
	res2 = 0    # partial result for the denominator in the formula
	
	# calculate the sums
	for i in range(len(y)):
		res1 += (y[i] - y_hat[i]) ** 2
		res2 += (y[i] - y_mean) ** 2
	
	# calculate the R^2 coefficient
	result = 1 - (res1 / res2)
	return result
\end{python}

Implementujeme metódu na dátach \verb|A04wine.csv|. Načítame dáta pomocou \pyth|pandas.read_csv()|, rozdelíme ich do premenných (rovnako ako v predošlých úlohách):

\begin{python}
	data = pd.read_csv('data/A04wine.csv')
	y = data['Price']
	x = data[['WinterRain', 'AGST', 'HarvestRain', 'Age', 'FrancePop']]
\end{python}

Podobne ako vyššie, vyriešime potrebné LP problémy pre načítané dáta a vypočítame $R^2$ koeficient:

\begin{python}
	betas = solve.x[:numberOfVariablesBeta]
	betas_inf = solve_inf.x[:numberOfVariablesBeta]
	
	r_squared(x, y, betas)
	r_squared(x, y, betas_inf)
\end{python}
	
Vypočítané príslušné R-kvadráty teda sú:

\begin{align*}
	R^{2}_{(1)} &\approx  0.78813 \\
	R^{2}_{(\infty)} &\approx 0.80649
\end{align*}

Z toho môžeme usúdiť, že pre dané dáta regresia pomocou Chebyshevovej normy lepšie zachytáva rozptyl dát.

\end{document}
