\documentclass[12pt,oneside,a4paper,slovak]{article}           

\usepackage{latexsym}
\usepackage{amsfonts}
\usepackage{amsthm}
\usepackage{amstext}
\usepackage{amsmath}
\usepackage{amscd}
\usepackage{amssymb}
\usepackage{graphicx}
\usepackage{titlesec}
\usepackage{listings}
\usepackage{pythonhighlight}
\usepackage[skip=0pt]{caption}
\usepackage{geometry}
\usepackage[final]{hyperref}
\usepackage[T1]{fontenc}                % slovencina
\usepackage[utf8]{inputenc}
\usepackage[slovak]{babel}

\geometry{top=2cm, bottom=2cm, left=2.5cm, right=2.5cm, headheight=0.6cm, headsep=0.3cm, footskip=1cm}

\renewcommand{\rmdefault}{phv} % Arial
\renewcommand{\sfdefault}{phv} % Arial
\renewcommand{\baselinestretch}{1}
\newcommand*{\vertbar}{\rule[0ex]{0.3pt}{1.5ex}}
\newcommand*{\horzbar}{\rule[0ex]{1.5ex}{0.3pt}}


\begin{document}   

\section*{Úloha D}
Do triedy \pyth|Model| sme pridali metódu na výpočet R-kvadrátu podľa definície :\\
\begin{math}
R^2 = 1 - \frac{\sum_{i=1}^{n} (y_i - \hat{y}_i)^2}{\sum_{i=1}^{n} (y_i - \bar{y})^2}
\end{math}
\\
\begin{python}
def r2(self) -> float:

        if len(self.beta) == 0:
            print('Model is not solved')
            return 0.0

        y_hat = self.beta[0] + np.dot(self.x_vect.transpose(), self.beta[1:])
        y_mean = np.mean(self.y)

        res1 = 0
        res2 = 0

        for i in range(len(self.y)):
            res1 += (self.y[i] - y_hat[i]) ** 2
            res2 += (self.y[i] - y_mean) ** 2

        result = 1 - (res1 / res2)
        return result
\end{python}
Implementujeme metódu na dátach \pyth|A04wine.csv| :
\begin{itemize}
	\item načítame dáta pomocou \pyth|pandas|, rozdelíme ich do premenných (rovnako ako v predošlých úlohách)
	\begin{python}
	data = pd.read_csv('data/A04wine.csv')
	y = data['Price']
	x = data[['WinterRain', 'AGST', 'HarvestRain', 'Age', 'FrancePop']]
	x = x.to_numpy().transpose()
	\end{python}
	\item naimportujeme predom zadefinované \pyth|L1| a \pyth|LInf| modely
	\item zostavíme LP problémy cez obe modely pre načítané dáta a vyriešime ich
	\begin{python}
	# utilize developed L1 and LInf regression model classes
	l1_model = L1Model(y, x)
	linf_model = LInfModel(y, x)
	
	# solve LP problems
	l1_model.solve()
	linf_model.solve()
	\end{python}
	\item po vyriešení zavoláme na dané modely metódu \pyth|r2()|, čím získame príslušné R-kvadráty
	\begin{python}
	# calculate R-squared coefficient
	print(f'R-squared for L1 regression on wine data: {l1_model.r2()}')      
	print(f'R-squared for LInf regression on wine data: {linf_model.r2()}')   
	\end{python}
\end{itemize}
Podľa modelov sme dostali nasledujúce výsledky:
\\
\begin{equation*}R^{2}_{l_1} \approx  0.78813\end{equation*}
\begin{equation*}R^{2}_{l_{\infty}} \approx 0.80649\end{equation*}
\\
Z čoho vieme povedať, že pre dané dáta je pre predikovanie o trochu presnejšia Chebyshevova norma.



\end{document}
