\documentclass[report.tex]{subfiles}

\begin{document}
\section{Úloha A}\label{sec:A}

Máme dané vektory $y, x_1, x_2, \dots, x_k$. Chceme nájsť parametre $\beta_0, \beta_1,\dots,\beta_k$ také, aby pre vektor $\hat{y} = \beta_0 + \beta_1x_1 + \dots + \beta_kx_k$, boli normy $||y - \hat{y}||_1$ a $||y - \hat{y}||_{\infty}$ minimálne. 

Vyjadrime vektor $\hat{y}$ ako súčin matice a vektora $\beta = (\beta_0, \beta_1, \dots, \beta_k)^T$.

\begin{align*}
	\hat{y} = \beta_0 + \beta_1x_1 + \dots + \beta_kx_k = 
	\left(
		\begin{array}{ccccc}
			\vertbar & \vertbar & \vertbar &  & \vertbar \\
			\mathbf{1}_n & x_1 & x_2 & \dots & x_k \\
			\vertbar & \vertbar & \vertbar &  & \vertbar
		\end{array}
	\right)
	\beta
	=\colon
	\mathbf{A} \beta
\end{align*}

\subsection{Minimalizovanie $L^1$ normy}

Prevedieme problém zo zadania do tvaru:

\begin{align*}
	\text{min}~ &c^Tx\\
	&Ax \geq b
\end{align*}

Zaveďme si nový vektor premenných $t \in \mathbb{R}^n$, ktorým ohraničíme normu $||y - \mathbf{A} \beta||_1$.

\begin{align*}
	-t \leq y - \mathbf{A} \beta &\leq t
\end{align*}	

Pre obe ohraničenia, odseparujme premenné od konštánt a prevedieme do maticového tvaru.

\begin{align*}
	\left(
		\begin{array}{c|c}
			\mathbf{A} & \mathbb{I}_n
		\end{array}
	\right)
	\left(
		\begin{array}{c}
			\beta \\
			\hline
			t
		\end{array}
	\right) &\geq y \\
	\left(
		\begin{array}{c|c}
			-\mathbf{A} & \mathbb{I}_n
		\end{array}
	\right)
	\left(
		\begin{array}{c}
			\beta \\
			\hline
			t
		\end{array}
	\right) &\geq -y
\end{align*}

Minimalizovanie $L^{1}$ normy ako úloha lineárneho programovania vyzerá teda nasledovne.

\begin{align}
	\text{min}~ &
	\left(
		\begin{array}{c|c}
			\mathbf{0}_{k+1}^T & \mathbf{1}_n^T
		\end{array}
	\right)
	\left(
		\begin{array}{c}
			\beta \\
			\hline
			t
		\end{array}
	\right) \nonumber \\
	&\left(
		\begin{array}{c|c}
			\mathbf{A} & \mathbb{I}_n \\
			\hline
			-\mathbf{A} & \mathbb{I}_n
		\end{array}
	\right)
	\left(
		\begin{array}{c}
			\beta \\
			\hline
			t
		\end{array}
	\right)
	\geq
	\left(
		\begin{array}{c}
			y \\
			\hline
			-y
		\end{array}
	\right) \label{P1} \\
	&\beta \in \mathbb{R}^{k+1},~t \geq \mathbf{0}_{n} \nonumber
\end{align}

\subsubsection{Prípustnosť a optimalita}\label{sec:1Optim}

Dokážme, že \eqref{P1} je úloha, ktorá nadobúda optimálne riešenie pre ľubovoľné vektory $y, x_1, x_2, \dots, x_k$. Nech $|y| := (|y_1|, |y_2|, \dots, |y_n|)^T$ pre $y = (y_1, y_2, \dots, y_n)^T$. Ukážme prípustnosť zvolením $\beta = \mathbf{0}_{k+1}$ a $t = |y|$:

\begin{align*}
	\left(
		\begin{array}{c|c}
			\mathbf{A} & \mathbb{I}_n \\
			\hline
			-\mathbf{A} & \mathbb{I}_n
		\end{array}
	\right)
	\left(
		\begin{array}{c}
			\mathbf{0}_{k+1} \\
			\hline
			|y|
		\end{array}
	\right)
	=
	\left(
		\begin{array}{c}
			|y| \\
			\hline
			|y|
		\end{array}
	\right)
	&\geq
	\left(
		\begin{array}{c}
			y \\
			\hline
			-y
		\end{array}
	\right) \\
	|y| &\geq \mathbf{0}_{n} 
\end{align*}

\newpage

Vidíme, že obe ohraničenia platia, čiže $\left(\mathbf{0}_{k+1}^T, |y|^T \right)^T$ je prípustné riešenie.

Optimalitu ukážeme zo silnej duality. Sformulujme duálnu úlohu pre duálne premenné $\alpha_1, \alpha_2 \in \mathbb{R}^{n}$:

\begin{align*}
	\text{max}~ &
	\left(
		\begin{array}{c|c}
			y^T & -y^T
		\end{array}
	\right)
	\left(
		\begin{array}{c}
			\alpha_1 \\
			\hline
			\alpha_2
		\end{array}
	\right) \\
	&\left(
		\begin{array}{c|c}
			\mathbf{A}^T & -\mathbf{A}^T
		\end{array}
	\right)
	\left(
		\begin{array}{c}
			\alpha_1 \\
			\hline
			\alpha_2
		\end{array}
	\right)
	=
	\mathbf{0}_{k+1} \\
	&\left(
		\begin{array}{c|c}
			\mathbb{I}_n & \mathbb{I}_n
		\end{array}
	\right)
	\left(
		\begin{array}{c}
			\alpha_1 \\
			\hline
			\alpha_2
		\end{array}
	\right)
	\leq
	\mathbf{1}_{n} \\
	&\alpha_1, \alpha_2 \geq \mathbf{0}_{n} 
\end{align*}

Vidíme, že táto úloha je prípustná pre $\alpha_1 = \alpha_2 = \mathbf{0}_n$. Z prípustnosti primárnej a duálnej úlohy teda vyplýva, že úloha \eqref{P1} nadobúda optimálne riešenie pre ľubovoľnú voľbu počiatočných vektorov.

\subsection{Minimalizovanie $L^{\infty}$ normy}

Budeme používať rovnaké značenie pre predikovaný vektor hodnôt $\hat{y} = \mathbf{A}\beta$ ako pri formulácii $L^1$ normy. Zaveďme si skalár $\gamma \in \mathbb{R}$, ktorým ohraničíme normu $||y - \mathbf{A} \beta||_{\infty}$.

\begin{align*}
	-\gamma \mathbf{1}_n \leq y - \mathbf{A} \beta &\leq \gamma \mathbf{1}_n
\end{align*}

Pre jednotlivé ohraničenia odseparujeme premenné od konštánt a zapíšeme v maticovom tvare.

\begin{align*}
	\left(
		\begin{array}{c|c}
			\mathbf{A} & \mathbf{1}_n
		\end{array}
	\right)
	\left(
		\begin{array}{c}
			\beta \\
			\hline
			\gamma
		\end{array}
	\right) & \geq y \\
	\left(
		\begin{array}{c|c}
			-\mathbf{A} & \mathbf{1}_n
		\end{array}
	\right)
	\left(
		\begin{array}{c}
			\beta \\
			\hline
			\gamma
		\end{array}
	\right) &\geq -y \\
\end{align*}


Minimalizovanie $L^{\infty}$ normy ako úloha lineárneho programovania vyzerá teda nasledovne.

\begin{align}
	\text{min}~ &
	\left(
		\begin{array}{c|c}
			\mathbf{0}_{k+1}^T & 1
		\end{array}
	\right)
	\left(
		\begin{array}{c}
			\beta \\
			\hline
			\gamma
		\end{array}
	\right) \nonumber \\
	&\left(
		\begin{array}{c|c}
			\mathbf{A} & \mathbf{1}_n \\
			\hline
			-\mathbf{A} & \mathbf{1}_n
		\end{array}
	\right)
	\left(
		\begin{array}{c}
			\beta \\
			\hline
			\gamma
		\end{array}
	\right)
	\geq
	\left(
		\begin{array}{c}
			y \\
			\hline
			-y
		\end{array}
	\right) \label{Pinf}\\
	&\beta \in \mathbb{R}^{k+1},~\gamma \geq 0 \nonumber
\end{align}

\newpage

\subsubsection{Prípustnosť a optimalita}\label{sec:InfOptim}

Podobný spôsobom ako vyššie ukážeme optimalitu \eqref{Pinf}. Nech $\beta = \mathbf{0}_{k+1}$ a $\gamma = |\tilde{y}|$, kde $|\tilde{y}| := |\max(y_1, y_2, \dots, y_n)|$ pre $y = (y_1, y_2, \dots, y_n)^T$:

\begin{align*}
	\left(
	\begin{array}{c|c}
		\mathbf{A} & \mathbf{1}_n \\
		\hline
		-\mathbf{A} & \mathbf{1}_n
	\end{array}
	\right)
	\left(
	\begin{array}{c}
		\mathbf{0}_{k+1} \\
		\hline
		|\tilde{y}|
	\end{array}
	\right)
	=
	\left(
		\begin{array}{c}
			|\tilde{y}| \mathbf{1}_n \\
			\hline
			|\tilde{y}| \mathbf{1}_n
		\end{array}
	\right)
	&\geq
	\left(
		\begin{array}{c}
			y \\
			\hline
			-y
		\end{array}
	\right) \\
	|\tilde{y}| &\geq 0 
\end{align*}

Obe ohraničenia platia, čiže $(\mathbf{0}_{k+1}^T, |\tilde{y}|)^T$ je prípustné riešenie. Sformulujme duálnu úlohu s duálnymi premennými $\alpha_1, \alpha_2 \in \mathbb{R}^n$:

\begin{align*}
	\text{max}~ &
	\left(
		\begin{array}{c|c}
			y^T & -y^T
		\end{array}
	\right)
	\left(
		\begin{array}{c}
			\alpha_1 \\
			\hline
			\alpha_2
		\end{array}
	\right) \\
	&\left(
		\begin{array}{c|c}
			\mathbf{A}^T & -\mathbf{A}^T
		\end{array}
	\right)
	\left(
		\begin{array}{c}
			\alpha_1 \\
			\hline
			\alpha_2
		\end{array}
	\right)
	=
	\mathbf{0}_{k+1} \\
	&\left(
		\begin{array}{c|c}
			\mathbf{1}_n^T & \mathbf{1}_n^T
		\end{array}
	\right)
	\left(
		\begin{array}{c}
			\alpha_1 \\
			\hline
			\alpha_2
		\end{array}
	\right)
	\leq
	1 \\
	&\alpha_1, \alpha_2 \geq \mathbf{0}_{n} 
\end{align*}

Rovnako vidíme, že táto úloha je prípustná pre $\alpha_1 = \alpha_2 = \mathbf{0}_n$. Teda, zo silnej duality, úloha \eqref{Pinf} nadobúda optimálne riešenie pre ľubovoľnú voľbu počiatočných vektorov.

\end{document}