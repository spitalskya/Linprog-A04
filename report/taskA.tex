\documentclass[12pt,oneside,a4paper,slovak]{article}           

\usepackage{latexsym}
\usepackage{amsfonts}
\usepackage{amsthm}
\usepackage{amstext}
\usepackage{amsmath}
\usepackage{amscd}
\usepackage{amssymb}
\usepackage{graphicx}
\usepackage{titlesec}
\usepackage{listings}
\usepackage{pythonhighlight}
\usepackage[skip=0pt]{caption}
\usepackage{geometry}
\usepackage[final]{hyperref}
\usepackage[T1]{fontenc}                % slovencina
\usepackage[utf8]{inputenc}
\usepackage[slovak]{babel}

\geometry{top=2cm, bottom=2cm, left=2.5cm, right=2.5cm, headheight=0.6cm, headsep=0.3cm, footskip=1cm}

\renewcommand{\rmdefault}{phv} % Arial
\renewcommand{\sfdefault}{phv} % Arial
\renewcommand{\baselinestretch}{1}
\newcommand*{\vertbar}{\rule[0ex]{0.3pt}{1.5ex}}
\newcommand*{\horzbar}{\rule[0ex]{1.5ex}{0.3pt}}


\begin{document}   

\section*{Úloha A}

Máme dané vektory $y, x_1, x_2, \dots, x_k$. Chceme nájsť parametre $\beta_0, \beta_1,\dots,\beta_k$ také, aby pre vektor $\hat{y} = \beta_0 + \beta_1x_1 + \dots + \beta_kx_k$, boli normy $||y - \hat{y}||_1$ a $||y - \hat{y}||_{\infty}$ minimálne. 

Vyjadrime vektor $\hat{y}$ ako súčin matice a vektora $\beta = (\beta_0, \beta_1, \dots, \beta_k)$.

\begin{align*}
	\hat{y} = \beta_0 + \beta_1x_1 + \dots + \beta_kx_k = 
	\left(
		\begin{array}{ccccc}
			\vertbar & \vertbar & \vertbar &  & \vertbar \\
			\mathbf{1}_n & x_1 & x_2 & \dots & x_k \\
			\vertbar & \vertbar & \vertbar &  & \vertbar
		\end{array}
	\right)
	\beta
	=\colon
	\mathbf{A} \beta
\end{align*}

\subsection*{Minimalizovanie $L^1$ normy}

Prevedieme problém zo zadania do tvaru:

\begin{align*}
	\text{min}~ &c^Tx\\
	Ax &\leq b
\end{align*}

Tento tvar vyžaduje metóda \pyth|linprog| z knižnice \pyth|scipy.optimize| pre Python.

Zaveďme si nový vektor premenných $t \in \mathbb{R}^n$, ktorým ohraničíme normu $||y - \mathbf{A} \beta||_1$.

\begin{align*}
	-t \leq y - \mathbf{A} \beta &\leq t
\end{align*}	

Pre prvé ohraničenie, odseparujme premenné od konštánt a preveďme do maticového tvaru.

\begin{align*}
	- \mathbf{A} \beta - t &\leq -y \\
	\left(
		\begin{array}{c|c}
			-\mathbf{A} & -\mathbb{I}_n
		\end{array}
	\right)
	\left(
		\begin{array}{c}
			\boldsymbol{\beta} \\
			\hline
			t
		\end{array}
	\right) &\leq -y
\end{align*}

Podobne pre druhé ohraničenie.

\begin{align*}
	\mathbf{A} \beta - t &\leq y \\
	\left(
		\begin{array}{c|c}
			\mathbf{A} & -\mathbb{I}_n
		\end{array}
	\right)
	\left(
		\begin{array}{c}
			\beta \\
			\hline
			t
		\end{array}
	\right) &\leq y
\end{align*}

Minimalizovanie $L^{1}$ normy ako úloha lineárneho programovania vyzerá teda nasledovne.

\begin{align*}
	\text{min}~ &
	\left(
		\begin{array}{c|c}
			\mathbf{0}_{k+1} & \mathbf{1}_n
		\end{array}
	\right)
	\left(
		\begin{array}{c}
			\beta \\
			\hline
			t
		\end{array}
	\right) \\
	&\left(
		\begin{array}{c|c}
			-\mathbf{A} & -\mathbb{I}_n \\
			\hline
			\mathbf{A} & -\mathbb{I}_n
		\end{array}
	\right)
	\left(
		\begin{array}{c}
			\beta \\
			\hline
			t
		\end{array}
	\right)
	\leq
	\left(
		\begin{array}{c}
			-y \\
			\hline
			y
		\end{array}
	\right)
\end{align*}

\newpage

\subsection*{Minimalizovanie $L^{\infty}$ normy}

Budeme používať podobné značenie ako pri formulácii $L^1$ normy. Zaveďme si skalár $\gamma \in \mathbb{R}$, ktorým ohraničíme normu $||y - \mathbf{A} \beta||_{\infty}$.

\begin{align*}
	-\gamma \mathbf{1}_n \leq y - \mathbf{A} \beta &\leq \gamma \mathbf{1}_n
\end{align*}

Pre jednotlivé ohraničenia odseparujeme premenné od konštánt a zapíšeme v maticovom tvare.

\begin{align*}
	\left(
		\begin{array}{c|c}
			-\mathbf{A} & -\mathbf{1}_n
		\end{array}
	\right)
	\left(
		\begin{array}{c}
			\beta \\
			\hline
			\gamma
		\end{array}
	\right) &\leq -y \\
	\left(
		\begin{array}{c|c}
			\mathbf{A} & -\mathbf{1}_n
		\end{array}
	\right)
	\left(
		\begin{array}{c}
			\beta \\
			\hline
			\gamma
		\end{array}
	\right) &\leq y \\
\end{align*}

Minimalizovanie $L^{\infty}$ normy ako úloha lineárneho programovania vyzerá teda nasledovne.


\begin{align*}
	\text{min}~ &
	\left(
		\begin{array}{c|c}
			\mathbf{0}_{k+1} & 1
		\end{array}
	\right)
	\left(
		\begin{array}{c}
			\beta \\
			\hline
			\gamma
		\end{array}
	\right) \\
	&\left(
		\begin{array}{c|c}
			-\mathbf{A} & -\mathbf{1}_n \\
			\hline
			\mathbf{A} & -\mathbf{1}_n
		\end{array}
	\right)
	\left(
		\begin{array}{c}
			\beta \\
			\hline
			\gamma
		\end{array}
	\right)
	\leq
	\left(
		\begin{array}{c}
			-y \\
			\hline
			y
		\end{array}
	\right)
\end{align*}



\end{document}